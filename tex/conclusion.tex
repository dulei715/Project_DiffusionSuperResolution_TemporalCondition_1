\section{Conclusion}
\label{sec:conclusion}

This thesis presents a comprehensive study on time series forecasting through the development of MRIC-Diff (Multi-Resolution Image-Enhanced Conditional Diffusion), a novel framework that addresses key limitations of existing methods by integrating time series-to-image transformation, multi-resolution trend decomposition, and hierarchical conditional diffusion modeling. The research makes significant contributions including: a delay embedding technique that converts variable-length time series into structured 2D image data while preserving temporal dependencies, enabling utilization of spatial inductive biases; a hierarchical trend decomposition module that explicitly models multi-scale temporal patterns (short-term fluctuations, seasonal cycles, and long-term trends); a hierarchical conditional diffusion model that performs denoising generation under multi-scale guidance, reducing generative randomness through coarse-to-fine constraints; and theoretical analyses (approximation ratio, time and space complexity) combined with empirical validation on real-world datasets demonstrating statistical consistency, computational feasibility, and 15-20\% average improvements in prediction accuracy over state-of-the-art methods. Future research directions include developing adaptive hyperparameter optimization mechanisms, incorporating external influencing factors, enhancing uncertainty quantification methods, improving computational efficiency through model compression or accelerated sampling, and exploring cross-domain generalization across healthcare, finance, and climate science applications. Overall, the MRIC-Diff framework advances time series forecasting by leveraging image representation, multi-resolution analysis, and conditional diffusion, providing both state-of-the-art performance on benchmarks and a new perspective for addressing fundamental challenges in temporal data modeling with significant potential for academic and industrial impact.
    